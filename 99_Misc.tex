\section{Introduction}\label{introduction}

\begin{frame}{Good Quote}

\begin{quote}
``All the cool kids use R.''

--- Anonymous
\end{quote}

\end{frame}

\begin{frame}{Introduction}

We need to cover a few additional topics, as well as review some topics:

\begin{enumerate}
\def\labelenumi{\arabic{enumi}.}
\tightlist
\item
  Loops (More Concrete Examples)
\item
  RMarkdown
\item
  Finding Relevant Packages
\item
  Modeling Summary
\item
  Final Quiz
\end{enumerate}

\end{frame}

\section{Loops}\label{loops}

\begin{frame}[fragile]{Loops}

This is a very important tool to use in data cleaning.

\vspace{25pt}

\begin{block}{Example: You have \texttt{999} as a placeholder throughout
your data. How can you quickly change those to \texttt{NA}?}

\end{block}

\end{frame}

\begin{frame}[fragile]{Loops}

Here's one way:

\begin{Shaded}
\begin{Highlighting}[]
\KeywordTok{library}\NormalTok{(furniture)}
\NormalTok{df[] <-}\StringTok{ }\KeywordTok{lapply}\NormalTok{(df, washer, }\DecValTok{999}\NormalTok{)}
\end{Highlighting}
\end{Shaded}

And another:

\begin{Shaded}
\begin{Highlighting}[]
\NormalTok{df[] <-}\StringTok{ }\KeywordTok{lapply}\NormalTok{(df, }\ControlFlowTok{function}\NormalTok{(x) }\KeywordTok{ifelse}\NormalTok{(x }\OperatorTok{==}\StringTok{ }\DecValTok{999}\NormalTok{, }\OtherTok{NA}\NormalTok{, x))}
\end{Highlighting}
\end{Shaded}

\vspace{20pt}

Note: the \texttt{{[}{]}} after the \texttt{df} forces \texttt{lapply()}
to keep it in a data.frame :)

\end{frame}

\begin{frame}{Loops}

\begin{block}{Another example: You need to recode the 5th, 10th, and
20th through 30th variables. How can you do that with a loop?}

\end{block}

\end{frame}

\begin{frame}[fragile]{Loops}

Here's one way:

\begin{Shaded}
\begin{Highlighting}[]
\KeywordTok{library}\NormalTok{(forcats)}
\NormalTok{df[, }\KeywordTok{c}\NormalTok{(}\DecValTok{5}\NormalTok{,}\DecValTok{10}\NormalTok{,}\DecValTok{20}\OperatorTok{:}\DecValTok{30}\NormalTok{)] <-}\StringTok{ }\KeywordTok{lapply}\NormalTok{(df[, }\KeywordTok{c}\NormalTok{(}\DecValTok{5}\NormalTok{,}\DecValTok{10}\NormalTok{,}\DecValTok{20}\OperatorTok{:}\DecValTok{30}\NormalTok{)], fct_recode, }\DecValTok{1}\NormalTok{ =}\StringTok{ "5"}\NormalTok{, }\DecValTok{2}\NormalTok{ =}\StringTok{ "4"}\NormalTok{, }\DecValTok{4}\NormalTok{ =}\StringTok{ "2"}\NormalTok{, }\DecValTok{5}\NormalTok{ =}\StringTok{ "1"}\NormalTok{)}
\end{Highlighting}
\end{Shaded}

And another:

\begin{Shaded}
\begin{Highlighting}[]
\KeywordTok{library}\NormalTok{(forcats)}
\ControlFlowTok{for}\NormalTok{ (i }\ControlFlowTok{in} \KeywordTok{c}\NormalTok{(}\DecValTok{5}\NormalTok{,}\DecValTok{10}\NormalTok{,}\DecValTok{20}\OperatorTok{:}\DecValTok{30}\NormalTok{))\{}
\NormalTok{  df[, i] =}\StringTok{ }\KeywordTok{fct_recode}\NormalTok{(df[, i], }\DecValTok{1}\NormalTok{ =}\StringTok{ "5"}\NormalTok{, }\DecValTok{2}\NormalTok{ =}\StringTok{ "4"}\NormalTok{, }\DecValTok{4}\NormalTok{ =}\StringTok{ "2"}\NormalTok{, }\DecValTok{5}\NormalTok{ =}\StringTok{ "1"}\NormalTok{)}
\NormalTok{\}}
\end{Highlighting}
\end{Shaded}

\end{frame}

\section{RMarkdown}\label{rmarkdown}

\begin{frame}[fragile]{RMarkdown}

So much to do! But focus on the basics :)

\begin{enumerate}
\def\labelenumi{\arabic{enumi}.}
\tightlist
\item
  The key to the whole thing is that you can use regular text and
  \texttt{R} code together.
\item
  The \texttt{R\ chunks} are very flexible so that you usually can get
  the results to look the way you want.
\item
  Markdown is a markup language but is not a WYSIWYG.\footnote<.->{What
    You See Is What You Get} But, you can always knit to see how it is
  looking.
\item
  Practice with it. Try things out.
\end{enumerate}

\end{frame}

\begin{frame}[fragile]{RMarkdown}

The text can be any text you would normally put in a document.

Code chunks can be any functioning \texttt{R} code.

Check out the following links for more information:

\begin{enumerate}
\def\labelenumi{\arabic{enumi}.}
\tightlist
\item
  \href{http://rmarkdown.rstudio.com/lesson-1.html}{RStudio and
  RMarkdown}
\item
  \href{https://www.youtube.com/watch?v=DNS7i2m4sB0}{Nice Youtube
  Tutorial}
\item
  \href{https://www.youtube.com/watch?v=cWJzjHh_3kk}{Other Nice, More
  Indepth Tutorial}
\end{enumerate}

\end{frame}

\section{Finding Relevant Packages}\label{finding-relevant-packages}

\begin{frame}{Finding Relevant Packages}

Much of this results from two things:

\begin{enumerate}
\def\labelenumi{\arabic{enumi}.}
\tightlist
\item
  Experience: you find packages that you like and trust over time
\item
  Google: The top results in Google are often good ones (at least well
  used).
\end{enumerate}

\end{frame}

\begin{frame}{Finding Cronbach's Alpha}

Example of finding a package for cronbach's alpha.

\end{frame}

\section{Modeling Summary}\label{modeling-summary}

\begin{frame}{Modeling Summary}

I wanted to provide you with a quick summary of packages and functions
needed for different model types.

\begin{longtable}[]{@{}lll@{}}
\toprule
Model & Package & Function\tabularnewline
\midrule
\endhead
T-Test & & t.test()\tabularnewline
ANOVA & & aov()\tabularnewline
Linear & & lm()\tabularnewline
Logistic & & glm() with `binomial'\tabularnewline
Poisson & & glm() with `poisson'\tabularnewline
GEE & gee or geepack & gee() or geeglm()\tabularnewline
Mixed Effects (MLM) & lme4 & lmer()\tabularnewline
SEM & lavaan & sem() or cfa()\tabularnewline
\bottomrule
\end{longtable}

\end{frame}

\section{Final Quiz}\label{final-quiz}

\begin{frame}{Final Quiz}

Let's see if you are comfortable with the following scenarios/lines of
code.

\end{frame}

\begin{frame}{1. Data Manipulation}

You want to do three things:

\begin{enumerate}
\def\labelenumi{\arabic{enumi}.}
\tightlist
\item
  Create a binary depression variable based on the level of the
  continuous depression variable (above a value of 10),
\item
  Filter out those that have no depression (level 0 of the new
  depression variable),
\item
  Find the mean, sd, and range of the productivity variable by sex.
\end{enumerate}

\end{frame}

\begin{frame}{2. Reshape That}

Your data is in wide format. It is a longitudinal data set, with two
observations per individual. What steps do you take to reshape it into
long format?

\end{frame}

\begin{frame}[fragile]{3. Visualize It}

Your data is again in wide format. It is longitudinal, with two
observations per individual.

You want to create:

\begin{enumerate}
\def\labelenumi{\arabic{enumi}.}
\tightlist
\item
  A scatterplot comparing mental aptitude and productivity at time one,
\item
  A line graph showing the means of productivity by sex at each time
  point,
\item
  Someone sends you the following code and you want to know what it
  does.
\end{enumerate}

\begin{Shaded}
\begin{Highlighting}[]
\KeywordTok{ggplot}\NormalTok{(df, }\KeywordTok{aes}\NormalTok{(}\DataTypeTok{x =}\NormalTok{ Sex, }\DataTypeTok{y =}\NormalTok{ Productivity, }\DataTypeTok{color =}\NormalTok{ Sex)) }\OperatorTok{+}
\StringTok{  }\KeywordTok{geom_boxplot}\NormalTok{() }\OperatorTok{+}
\StringTok{  }\KeywordTok{scale_color_manual}\NormalTok{(}\DataTypeTok{values =} \KeywordTok{c}\NormalTok{(}\StringTok{"red"}\NormalTok{, }\StringTok{"not_red"}\NormalTok{)) }\OperatorTok{+}
\StringTok{  }\KeywordTok{theme}\NormalTok{(}\DataTypeTok{legend.position =} \StringTok{"bottom"}\NormalTok{)}
\end{Highlighting}
\end{Shaded}

\end{frame}

\begin{frame}{4. Strut Yo' Stuff (i.e.~Modeling)}

If you have repeated measures data, what format does you data need to be
in to analyze it with either RM-ANOVA or Mixed Effects?

\end{frame}

\begin{frame}[fragile]{5. RMarkdown}

How do you create a new \texttt{rmarkdown} file?

In \texttt{rmarkdown} how can you insert inline r syntax?

\end{frame}

\section{Closing Remarks}\label{closing-remarks}

\begin{frame}[fragile]{Thanks!}

Thanks for participating. I hope this was a great start for you in using
\texttt{R} throughout your careers.

Let me know if I can help in the future with \texttt{R} or stat related
things.

\end{frame}
