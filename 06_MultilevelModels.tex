\section{Introduction}\label{introduction}

\begin{frame}{Good Quote}

\begin{quote}
``Simplicity does not precede complexity, but follows it.''

--- Alan Perlis
\end{quote}

\end{frame}

\begin{frame}{Introduction}

Multilevel data are more complex and don't meet the assumptions of
regular linear or generalized linear models. But with the right modeling
schemes, the results can be very interpretable and actionable.

Two powerful forms of multilevel modeling are:

\begin{enumerate}
\def\labelenumi{\arabic{enumi}.}
\tightlist
\item
  Generalized Estimating Equations (GEE)
\item
  Mixed effects (ME; i.e., hierarchical linear modeling, multilevel
  modeling)
\end{enumerate}

\end{frame}

\begin{frame}{GEE and ME}

Similarities:

\begin{itemize}
\tightlist
\item
  they both attempt to control for the lack of independence within
  clusters, although they do it in different ways.
\item
  built on linear regression which makes them flexible and powerful at
  finding relationships in the data.
\end{itemize}

\end{frame}

\begin{frame}{GEE and ME}

Differences:

\begin{itemize}
\tightlist
\item
  The interpretation is somewhat different between the two.
\item
  GEE is a population-averaged (e.g., marginal) model whereas ME is
  subject specific. In other words, \emph{GEE is the average effect}
  while \emph{ME is the effect found in the average person}.
\item
  In a linear model, these coefficients are the same but when we use
  different forms such as logistic or poisson, these can be quite
  different (although in my experience they generally tell a similar
  story). - ME models are much more complex than the GEE models and can
  struggle with convergence compared to the GEE.
\item
  This also means that GEE's are generally fitted much more quickly.
\end{itemize}

\end{frame}

\begin{frame}{GEE and ME}

Still the choice of the modeling technique should be driven by your
hypotheses and not totally dependent on speed of the computation.

\end{frame}

\begin{frame}{Prep the Data}

For both modeling techniques we want our data in long form.

\begin{itemize}
\tightlist
\item
  What this implies is that each row is an observation.
\item
  What this actually means about the data depends on the data.
\item
  For example, if you have repeated measures, then often data is stored
  in wide form---a row is an individual.
\item
  To make this long, we want each time point within a person to be a
  row---a single individual can have multiple rows but each row is a
  unique observation.
\end{itemize}

The NHANES data is in long form since we are working within community
clusters within this data. So, each row is an observation and each
cluster has multiple rows.

\end{frame}

\begin{frame}{Prep the Data}

\large

Note that although these analyses will be within community clusters
instead of within subjects (i.e.~repeated measures), the overall steps
will be the exact same.

\end{frame}

\begin{frame}{Reminder}

\large

This is not a multilevel modeling course. For this class we are only
going to demonstrate basic examples of it.

\end{frame}

\section{Generalized Estimating
Equations}\label{generalized-estimating-equations}

\begin{frame}[fragile]{GEE}

There are two packages, intimately related, that allow us to perform GEE
modeling

\begin{enumerate}
\def\labelenumi{\arabic{enumi}.}
\tightlist
\item
  \texttt{gee} and
\item
  \texttt{geepack}.
\end{enumerate}

These have some great features and make running a fairly complex model
pretty simple.

However, as great as they are, there are some annoying shortcomings.

\end{frame}

\begin{frame}{GEE}

GEE's, in general, want a few pieces of information from you.

\begin{enumerate}
\def\labelenumi{\arabic{enumi}.}
\tightlist
\item
  The outcome and predictors
\item
  A correlation structure
\item
  A variable that is cluster ID's.
\item
  The family (i.e.~the type of distribution).
\end{enumerate}

\emph{Since this is not longitudinal, but rather clustered within
communities, we'll assume for this analysis an unstructured correlation
structure. It is the most flexible and we have enough power for it
here.}

\end{frame}

\begin{frame}[fragile]{GEE}

For \texttt{geepack} to work, we need to filter out the missing values
for the variables that will be in the model.

\begin{Shaded}
\begin{Highlighting}[]
\NormalTok{df2 <-}\StringTok{ }\NormalTok{df }\OperatorTok
\StringTok{  }\KeywordTok{filter}\NormalTok{(}\KeywordTok{complete.cases}\NormalTok{(dep, famsize, sed, race, asthma))}
\end{Highlighting}
\end{Shaded}

\end{frame}

\begin{frame}{GEE}

We'll build the model with both packages (just for demonstration).

\end{frame}

\begin{frame}[fragile]{GEE}

\begin{Shaded}
\begin{Highlighting}[]
\KeywordTok{library}\NormalTok{(gee)}
\NormalTok{fit_gee <-}\StringTok{ }\KeywordTok{gee}\NormalTok{(dep }\OperatorTok{~}\StringTok{ }\NormalTok{asthma }\OperatorTok{+}\StringTok{ }\NormalTok{famsize }\OperatorTok{+}\StringTok{ }\NormalTok{sed }\OperatorTok{+}\StringTok{ }\NormalTok{race,}
               \DataTypeTok{data =}\NormalTok{ df2,}
               \DataTypeTok{id =}\NormalTok{ df2}\OperatorTok{$}\NormalTok{sdmvstra,}
               \DataTypeTok{corstr =} \StringTok{"unstructured"}\NormalTok{)}
\KeywordTok{summary}\NormalTok{(fit_gee)}\OperatorTok{$}\NormalTok{coef}

\KeywordTok{library}\NormalTok{(geepack)}
\NormalTok{fit_geeglm <-}\StringTok{ }\KeywordTok{geeglm}\NormalTok{(dep }\OperatorTok{~}\StringTok{ }\NormalTok{asthma }\OperatorTok{+}\StringTok{ }\NormalTok{famsize }\OperatorTok{+}\StringTok{ }\NormalTok{sed }\OperatorTok{+}\StringTok{ }\NormalTok{race,}
                     \DataTypeTok{data =}\NormalTok{ df2,}
                     \DataTypeTok{id =}\NormalTok{ df2}\OperatorTok{$}\NormalTok{sdmvstra,}
                     \DataTypeTok{corstr =} \StringTok{"unstructured"}\NormalTok{)}
\KeywordTok{summary}\NormalTok{(fit_geeglm)}
\end{Highlighting}
\end{Shaded}

\end{frame}

\begin{frame}[fragile]{GEE}

\tiny

\begin{verbatim}
##       (Intercept)      asthmaAsthma           famsize               sed 
##       2.500022059       1.356081567      -0.042132178       0.001362226 
## raceOtherHispanic         raceWhite         raceBlack         raceOther 
##       1.184995689       0.113949209       0.100536695      -0.555478773
\end{verbatim}

\begin{verbatim}
##                       Estimate   Naive S.E.    Naive z  Robust S.E.
## (Intercept)        2.495509790 0.2867816215  8.7017773 0.2690426648
## asthmaAsthma       1.353039007 0.1867101195  7.2467363 0.2137975620
## famsize           -0.039489294 0.0461945052 -0.8548483 0.0457474654
## sed                0.001358042 0.0003362291  4.0390382 0.0003551901
## raceOtherHispanic  1.192481318 0.3075562837  3.8772783 0.3309608614
## raceWhite          0.116185743 0.2531554533  0.4589502 0.2279687738
## raceBlack          0.096800821 0.2625826864  0.3686489 0.2360498473
## raceOther         -0.555053605 0.2809301544 -1.9757708 0.2406566044
##                     Robust z
## (Intercept)        9.2755169
## asthmaAsthma       6.3285989
## famsize           -0.8632018
## sed                3.8234244
## raceOtherHispanic  3.6030886
## raceWhite          0.5096564
## raceBlack          0.4100864
## raceOther         -2.3064133
\end{verbatim}

\end{frame}

\begin{frame}[fragile]{GEE}

\tiny

\begin{verbatim}
## 
## Call:
## geeglm(formula = dep ~ asthma + famsize + sed + race, data = df2, 
##     id = df2$sdmvstra, corstr = "unstructured")
## 
##  Coefficients:
##                     Estimate    Std.err   Wald Pr(>|W|)    
## (Intercept)        2.5579361  0.2700717 89.706  < 2e-16 ***
## asthmaAsthma       1.3492892  0.2156202 39.159 3.91e-10 ***
## famsize           -0.0446716  0.0457087  0.955 0.328415    
## sed                0.0013015  0.0003548 13.454 0.000244 ***
## raceOtherHispanic  1.1750373  0.3318983 12.534 0.000400 ***
## raceWhite          0.0806377  0.2295661  0.123 0.725392    
## raceBlack          0.0642028  0.2363255  0.074 0.785875    
## raceOther         -0.5902049  0.2413379  5.981 0.014463 *  
## ---
## Signif. codes:  0 '***' 0.001 '**' 0.01 '*' 0.05 '.' 0.1 ' ' 1
## 
## Estimated Scale Parameters:
##             Estimate Std.err
## (Intercept)    19.49  0.7843
## 
## Correlation: Structure = unstructured  Link = identity 
## 
## Estimated Correlation Parameters:
##           Estimate Std.err
## alpha.1:2  0.12480 0.01654
## alpha.1:3  0.42070 0.10339
## alpha.1:4  2.89640 1.06678
## alpha.1:5 -1.85447 0.20276
## alpha.2:3  0.12238 0.06330
## alpha.2:4 -0.08935 0.20229
## alpha.2:5  0.20541 0.03720
## alpha.3:4 -0.49597 0.11227
## alpha.3:5  0.25045 0.03879
## alpha.4:5 -0.66939 0.08761
## Number of clusters:   4109   Maximum cluster size: 5
\end{verbatim}

\end{frame}

\begin{frame}[fragile]{GEE}

The \texttt{gee} package doesn't directly provide p-values but provides
the z-scores, which can be used to find the p-values.

The \texttt{geepack} provides the p-values in the way you'll see in the
\texttt{lm()} and \texttt{glm()} functions.

These models are interpreted just as the regular GLM. It has adjusted
for the correlations within the clusters and provides valid standard
errors and p-values.

\end{frame}

\section{Mixed Effects}\label{mixed-effects}

\begin{frame}{Mixed Effects}

It is called ``mixed effects'' because we include both fixed and random
effects into the model simultaneously.

\begin{itemize}
\tightlist
\item
  Random effects are those that we don't necessarily care about the
  specific values but want to control for it and/or estimate the
  variance.
\item
  Fixed effects are those we are used to estimating in linear models and
  GLM's.
\end{itemize}

\end{frame}

\begin{frame}[fragile]{Mixed Effects}

These are a bit more clear with an example.

\begin{itemize}
\tightlist
\item
  We will do the same overall model as we did with the GEE but we'll use
  ME.
\item
  Use the \texttt{lme4} package
\end{itemize}

\footnotesize

\begin{Shaded}
\begin{Highlighting}[]
\KeywordTok{library}\NormalTok{(lme4)}
\NormalTok{fit_me <-}\StringTok{ }\KeywordTok{lmer}\NormalTok{(dep }\OperatorTok{~}\StringTok{ }\NormalTok{asthma }\OperatorTok{+}\StringTok{ }\NormalTok{famsize }\OperatorTok{+}\StringTok{ }\NormalTok{sed }\OperatorTok{+}\StringTok{ }\NormalTok{race }\OperatorTok{+}\StringTok{ }\NormalTok{(}\DecValTok{1} \OperatorTok{|}\StringTok{ }\NormalTok{cluster),}
               \DataTypeTok{data =}\NormalTok{ df2,}
               \DataTypeTok{REML =} \OtherTok{FALSE}\NormalTok{)}
\KeywordTok{summary}\NormalTok{(fit_me)}
\end{Highlighting}
\end{Shaded}

\end{frame}

\begin{frame}[fragile]{Mixed Effects}

\tiny

\begin{verbatim}
## Linear mixed model fit by maximum likelihood  ['lmerMod']
## Formula: dep ~ asthma + famsize + sed + race + (1 | cluster)
##    Data: df2
## 
##      AIC      BIC   logLik deviance df.resid 
##    25780    25844   -12880    25760     4427 
## 
## Scaled residuals: 
##    Min     1Q Median     3Q    Max 
## -1.327 -0.635 -0.355  0.272  5.435 
## 
## Random effects:
##  Groups   Name        Variance Std.Dev.
##  cluster  (Intercept)  0.105   0.324   
##  Residual             19.389   4.403   
## Number of obs: 4437, groups:  cluster, 14
## 
## Fixed effects:
##                    Estimate Std. Error t value
## (Intercept)        2.491678   0.302768    8.23
## asthmaAsthma       1.335445   0.186618    7.16
## famsize           -0.042857   0.046341   -0.92
## sed                0.001425   0.000337    4.23
## raceOtherHispanic  1.289890   0.320595    4.02
## raceWhite          0.008348   0.259449    0.03
## raceBlack          0.171658   0.273382    0.63
## raceOther         -0.552746   0.285512   -1.94
## 
## Correlation of Fixed Effects:
##             (Intr) asthmA famsiz sed    rcOthH racWht rcBlck
## asthmaAsthm -0.042                                          
## famsize     -0.510 -0.004                                   
## sed         -0.324 -0.044  0.051                            
## rcOthrHspnc -0.556 -0.032  0.051 -0.038                     
## raceWhite   -0.680 -0.038  0.135 -0.148  0.639              
## raceBlack   -0.643 -0.057  0.094 -0.131  0.624  0.775       
## raceOther   -0.580  0.000  0.048 -0.135  0.589  0.725  0.693
\end{verbatim}

\end{frame}

\begin{frame}[fragile]{Mixed Effects}

There are no p-values provided here. This is because degrees of freedom
are not well-defined in the ME framework.

A good way to test it can be through the \texttt{anova()} function,
comparing models. Let's compare a model with and without \texttt{asthma}
to see if the model is significantly better with it in.

\end{frame}

\begin{frame}[fragile]{Mixed Effects}

\footnotesize

\begin{Shaded}
\begin{Highlighting}[]
\NormalTok{fit_me1 <-}\StringTok{ }\KeywordTok{lmer}\NormalTok{(dep }\OperatorTok{~}\StringTok{ }\NormalTok{famsize }\OperatorTok{+}\StringTok{ }\NormalTok{sed }\OperatorTok{+}\StringTok{ }\NormalTok{race }\OperatorTok{+}\StringTok{ }\NormalTok{(}\DecValTok{1} \OperatorTok{|}\StringTok{ }\NormalTok{cluster),}
               \DataTypeTok{data =}\NormalTok{ df2,}
               \DataTypeTok{REML =} \OtherTok{FALSE}\NormalTok{)}

\KeywordTok{anova}\NormalTok{(fit_me, fit_me1)}
\end{Highlighting}
\end{Shaded}

\begin{verbatim}
## Data: df2
## Models:
## fit_me1: dep ~ famsize + sed + race + (1 | cluster)
## fit_me: dep ~ asthma + famsize + sed + race + (1 | cluster)
##         Df   AIC   BIC logLik deviance Chisq Chi Df Pr(>Chisq)    
## fit_me1  9 25829 25886 -12905    25811                            
## fit_me  10 25780 25844 -12880    25760  50.9      1    9.9e-13 ***
## ---
## Signif. codes:  0 '***' 0.001 '**' 0.01 '*' 0.05 '.' 0.1 ' ' 1
\end{verbatim}

\end{frame}

\begin{frame}[fragile]{Mixed Effects}

This comparison strongly suggests that \texttt{asthma} is a significant
predictor (\(\chi^2 = 50.5\), p \textless{} .001). We can do this with
both fixed and random effects, as below:

\footnotesize

\begin{Shaded}
\begin{Highlighting}[]
\NormalTok{fit_me2 <-}\StringTok{ }\KeywordTok{lmer}\NormalTok{(dep }\OperatorTok{~}\StringTok{ }\NormalTok{famsize }\OperatorTok{+}\StringTok{ }\NormalTok{sed }\OperatorTok{+}\StringTok{ }\NormalTok{race }\OperatorTok{+}\StringTok{ }\NormalTok{(}\DecValTok{1} \OperatorTok{|}\StringTok{ }\NormalTok{cluster),}
               \DataTypeTok{data =}\NormalTok{ df2,}
               \DataTypeTok{REML =} \OtherTok{TRUE}\NormalTok{)}
\NormalTok{fit_me3 <-}\StringTok{ }\KeywordTok{lmer}\NormalTok{(dep }\OperatorTok{~}\StringTok{ }\NormalTok{famsize }\OperatorTok{+}\StringTok{ }\NormalTok{sed }\OperatorTok{+}\StringTok{ }\NormalTok{race }\OperatorTok{+}\StringTok{ }\NormalTok{(}\DecValTok{1} \OperatorTok{+}\StringTok{ }\NormalTok{asthma }\OperatorTok{|}\StringTok{ }\NormalTok{cluster),}
               \DataTypeTok{data =}\NormalTok{ df2,}
               \DataTypeTok{REML =} \OtherTok{TRUE}\NormalTok{)}
\end{Highlighting}
\end{Shaded}

\end{frame}

\begin{frame}[fragile]{Mixed Effects}

\footnotesize

\begin{Shaded}
\begin{Highlighting}[]
\KeywordTok{anova}\NormalTok{(fit_me2, fit_me3, }\DataTypeTok{refit =} \OtherTok{FALSE}\NormalTok{)}
\end{Highlighting}
\end{Shaded}

\begin{verbatim}
## Data: df2
## Models:
## fit_me2: dep ~ famsize + sed + race + (1 | cluster)
## fit_me3: dep ~ famsize + sed + race + (1 + asthma | cluster)
##         Df   AIC   BIC logLik deviance Chisq Chi Df Pr(>Chisq)    
## fit_me2  9 25855 25912 -12918    25837                            
## fit_me3 11 25821 25892 -12900    25799  37.3      2      8e-09 ***
## ---
## Signif. codes:  0 '***' 0.001 '**' 0.01 '*' 0.05 '.' 0.1 ' ' 1
\end{verbatim}

Here, including random slopes for asthma appears to be significant
(\(\chi^2 = 36.9\), p \textless{} .001).

\end{frame}

\begin{frame}{Mixed Effects}

Linear mixed effects models converge pretty well. You'll see that the
conclusions and estimates are very similar to that of the GEE.

For generalized versions of ME, the convergence can be harder and more
picky. As we'll see below, it complains about large eigenvalues and
tells us to rescale some of the variables.

\end{frame}

\begin{frame}[fragile]{Generalized Mixed Effects}

\footnotesize

\begin{Shaded}
\begin{Highlighting}[]
\KeywordTok{library}\NormalTok{(lme4)}
\NormalTok{fit_gme <-}\StringTok{ }\KeywordTok{glmer}\NormalTok{(dep2 }\OperatorTok{~}\StringTok{ }\NormalTok{asthma }\OperatorTok{+}\StringTok{ }\NormalTok{famsize }\OperatorTok{+}\StringTok{ }\NormalTok{sed }\OperatorTok{+}\StringTok{ }\NormalTok{race }\OperatorTok{+}\StringTok{ }\NormalTok{(}\DecValTok{1} \OperatorTok{|}\StringTok{ }\NormalTok{cluster),}
                 \DataTypeTok{data =}\NormalTok{ df2,}
                 \DataTypeTok{family =} \StringTok{"binomial"}\NormalTok{)}
\end{Highlighting}
\end{Shaded}

\begin{verbatim}
## Warning in checkConv(attr(opt, "derivs"), opt$par, ctrl = control
## $checkConv, : Model failed to converge with max|grad| = 0.00854237 (tol =
## 0.001, component 1)
\end{verbatim}

\begin{verbatim}
## Warning in checkConv(attr(opt, "derivs"), opt$par, ctrl = control$checkConv, : Model is nearly unidentifiable: very large eigenvalue
##  - Rescale variables?;Model is nearly unidentifiable: large eigenvalue ratio
##  - Rescale variables?
\end{verbatim}

\end{frame}

\begin{frame}[fragile]{Generalized Mixed Effects}

\begin{block}{Warnings!}

\begin{itemize}
\tightlist
\item
  \texttt{sed} is huge compared to the other variables.
\item
  If we simply rescale it, using the \texttt{I()} function within the
  model formula, we can rescale it by 1,000. - Here, that is all it
  needed to converge.
\end{itemize}

\end{block}

\end{frame}

\begin{frame}[fragile]{Generalized Mixed Effects}

\footnotesize

\begin{Shaded}
\begin{Highlighting}[]
\KeywordTok{library}\NormalTok{(lme4)}
\NormalTok{fit_gme <-}\StringTok{ }\KeywordTok{glmer}\NormalTok{(dep2 }\OperatorTok{~}\StringTok{ }\NormalTok{asthma }\OperatorTok{+}\StringTok{ }\NormalTok{famsize }\OperatorTok{+}\StringTok{ }\KeywordTok{I}\NormalTok{(sed}\OperatorTok{/}\DecValTok{1000}\NormalTok{) }\OperatorTok{+}\StringTok{ }\NormalTok{race }\OperatorTok{+}\StringTok{ }\NormalTok{(}\DecValTok{1} \OperatorTok{|}\StringTok{ }\NormalTok{cluster),}
                 \DataTypeTok{data =}\NormalTok{ df2,}
                 \DataTypeTok{family =} \StringTok{"binomial"}\NormalTok{)}
\KeywordTok{summary}\NormalTok{(fit_gme)}
\end{Highlighting}
\end{Shaded}

\end{frame}

\begin{frame}[fragile]{Generalized Mixed Effects}

\tiny

\begin{verbatim}
## Generalized linear mixed model fit by maximum likelihood (Laplace
##   Approximation) [glmerMod]
##  Family: binomial  ( logit )
## Formula: dep2 ~ asthma + famsize + I(sed/1000) + race + (1 | cluster)
##    Data: df2
## 
##      AIC      BIC   logLik deviance df.resid 
##     2665     2722    -1323     2647     4428 
## 
## Scaled residuals: 
##    Min     1Q Median     3Q    Max 
## -0.635 -0.329 -0.295 -0.258  5.032 
## 
## Random effects:
##  Groups  Name        Variance Std.Dev.
##  cluster (Intercept) 0.0232   0.152   
## Number of obs: 4437, groups:  cluster, 14
## 
## Fixed effects:
##                   Estimate Std. Error z value Pr(>|z|)    
## (Intercept)        -2.6316     0.2435  -10.81  < 2e-16 ***
## asthmaAsthma        0.5619     0.1281    4.39  1.2e-05 ***
## famsize            -0.0336     0.0374   -0.90   0.3696    
## I(sed/1000)         0.5835     0.2618    2.23   0.0258 *  
## raceOtherHispanic   0.7564     0.2421    3.12   0.0018 ** 
## raceWhite           0.0955     0.2159    0.44   0.6581    
## raceBlack           0.0531     0.2277    0.23   0.8155    
## raceOther          -0.4950     0.2591   -1.91   0.0560 .  
## ---
## Signif. codes:  0 '***' 0.001 '**' 0.01 '*' 0.05 '.' 0.1 ' ' 1
## 
## Correlation of Fixed Effects:
##             (Intr) asthmA famsiz I(/100 rcOthH racWht rcBlck
## asthmaAsthm -0.057                                          
## famsize     -0.491 -0.012                                   
## I(sed/1000) -0.324 -0.042  0.031                            
## rcOthrHspnc -0.653 -0.031  0.044 -0.029                     
## raceWhite   -0.715 -0.037  0.132 -0.148  0.709              
## raceBlack   -0.684 -0.064  0.088 -0.124  0.715  0.781       
## raceOther   -0.571 -0.003  0.046 -0.122  0.606  0.688  0.653
\end{verbatim}

\end{frame}

\section{Conclusions}\label{conclusions}

\begin{frame}[fragile]{Conclusions}

This has been a really brief introduction into a thriving, large field
of statistical analyses. These are the general methods for using
\texttt{R} to analyze multilevel data. Our next chapter will discuss
more modeling techniques in \texttt{R}, including mediation, mixture,
and structural equation modeling.

\end{frame}
