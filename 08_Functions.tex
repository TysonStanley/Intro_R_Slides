\section{Introduction}\label{introduction}

\begin{frame}{Introduction}

\textbf{User-defined functions} --- functions you create yourself ---
are an important tool.

We will show how you can make your own functions.

\end{frame}

\begin{frame}[fragile]{What is a Function Anyway?}

It is a special \texttt{R} object that performs something on another
object or the environment.

For example,

\begin{itemize}
\tightlist
\item
  \texttt{ggplot()} takes a \texttt{data.frame} object and variable
  names and produces a plot.
\item
  \texttt{lm()} takes a formula, a \texttt{data.frame} and performs a
  statistical model.
\item
  \texttt{gather()} takes a \texttt{data.frame} and produces another
  \texttt{data.frame} in another form
\end{itemize}

\end{frame}

\section{Creating a Function}\label{creating-a-function}

\begin{frame}[fragile]{Creating a Function}

It all starts with the function \texttt{function()}.

\begin{Shaded}
\begin{Highlighting}[]
\NormalTok{myfunction <-}\StringTok{ }\ControlFlowTok{function}\NormalTok{(arguments)\{}
\NormalTok{  stuff =}\StringTok{ }\KeywordTok{that}\NormalTok{(you)}
\NormalTok{  want =}\StringTok{ }\NormalTok{it }\OperatorTok{+}\StringTok{ }\NormalTok{to }\OperatorTok{+}\StringTok{ }\NormalTok{do }\OperatorTok{+}\StringTok{ }\DecValTok{4}
\NormalTok{  you}
\NormalTok{\}}
\end{Highlighting}
\end{Shaded}

After \texttt{function()} we use \texttt{\{} and \texttt{\}}. Everything
in between is what you want the function to do.

All you need to do is run the function and you can use it in the
\texttt{R} session.

\end{frame}

\section{Named Functions}\label{named-functions}

\begin{frame}[fragile]{Named Functions}

These are functions that you assign a name to.

\begin{Shaded}
\begin{Highlighting}[]
\NormalTok{mean2 <-}\StringTok{ }\ControlFlowTok{function}\NormalTok{(x)\{}
\NormalTok{  n <-}\StringTok{ }\KeywordTok{length}\NormalTok{(x)}
\NormalTok{  m <-}\StringTok{ }\NormalTok{(}\DecValTok{1}\OperatorTok{/}\NormalTok{n) }\OperatorTok{*}\StringTok{ }\KeywordTok{sum}\NormalTok{(x)}
  \KeywordTok{return}\NormalTok{(m)}
\NormalTok{\}}
\end{Highlighting}
\end{Shaded}

Now, \texttt{mean2()} is a function you can use.

\emph{What does this function do?}

\end{frame}

\begin{frame}[fragile]{Named Functions}

\begin{Shaded}
\begin{Highlighting}[]
\NormalTok{v1 <-}\StringTok{ }\KeywordTok{c}\NormalTok{(}\DecValTok{1}\NormalTok{,}\DecValTok{3}\NormalTok{,}\DecValTok{2}\NormalTok{,}\DecValTok{4}\NormalTok{,}\DecValTok{2}\NormalTok{,}\DecValTok{1}\NormalTok{,}\DecValTok{2}\NormalTok{,}\DecValTok{1}\NormalTok{,}\DecValTok{1}\NormalTok{,}\DecValTok{1}\NormalTok{)   ## vector to try}
\end{Highlighting}
\end{Shaded}

Let's give it a try using the vector \texttt{v1}

\begin{Shaded}
\begin{Highlighting}[]
\KeywordTok{mean2}\NormalTok{(v1)                      ## our function}
\end{Highlighting}
\end{Shaded}

\begin{verbatim}
## [1] 1.8
\end{verbatim}

\begin{Shaded}
\begin{Highlighting}[]
\KeywordTok{mean}\NormalTok{(v1)                       ## the base R function}
\end{Highlighting}
\end{Shaded}

\begin{verbatim}
## [1] 1.8
\end{verbatim}

\end{frame}

\begin{frame}[fragile]{Named Functions}

For practice, we will write one more function. Let's make a function
that takes a vector and gives us the N, the mean, and the standard
deviation.

\begin{Shaded}
\begin{Highlighting}[]
\NormalTok{important_statistics <-}\StringTok{ }\ControlFlowTok{function}\NormalTok{(x, }\DataTypeTok{na.rm=}\OtherTok{FALSE}\NormalTok{)\{}
\NormalTok{  N  <-}\StringTok{ }\KeywordTok{length}\NormalTok{(x)}
\NormalTok{  M  <-}\StringTok{ }\KeywordTok{mean}\NormalTok{(x, }\DataTypeTok{na.rm=}\NormalTok{na.rm)}
\NormalTok{  SD <-}\StringTok{ }\KeywordTok{sd}\NormalTok{(x, }\DataTypeTok{na.rm=}\NormalTok{na.rm)}
  
\NormalTok{  final <-}\StringTok{ }\KeywordTok{c}\NormalTok{(N, M, SD)}
  \KeywordTok{return}\NormalTok{(final)}
\NormalTok{\}}
\end{Highlighting}
\end{Shaded}

\end{frame}

\begin{frame}[fragile]{Named functions}

One of the first things you should note is that we included a second
argument in the function seen as \texttt{na.rm=FALSE} (you can have as
many arguments as you want within reason).

\begin{itemize}
\tightlist
\item
  This argument has a default that we provide as \texttt{FALSE}
\item
  We take what is provided in the \texttt{na.rm} and give that to both
  the \texttt{mean()} and \texttt{sd()} functions.
\item
  Finally, you should notice that we took several pieces of information
  and combined them into the \texttt{final} object and returned that.
\end{itemize}

\end{frame}

\begin{frame}[fragile]{Named Functions}

Let's try it out with the vector we created earlier.

\begin{Shaded}
\begin{Highlighting}[]
\KeywordTok{important_statistics}\NormalTok{(v1)}
\end{Highlighting}
\end{Shaded}

\begin{verbatim}
## [1] 10.000000  1.800000  1.032796
\end{verbatim}

\end{frame}

\begin{frame}[fragile]{Named Functions}

Looks good but we may want to change a few aesthetics. In the following
code, we adjust it so we have each one labeled.

\begin{Shaded}
\begin{Highlighting}[]
\NormalTok{important_statistics2 <-}\StringTok{ }\ControlFlowTok{function}\NormalTok{(x, }\DataTypeTok{na.rm=}\OtherTok{FALSE}\NormalTok{)\{}
\NormalTok{  N  <-}\StringTok{ }\KeywordTok{length}\NormalTok{(x)}
\NormalTok{  M  <-}\StringTok{ }\KeywordTok{mean}\NormalTok{(x, }\DataTypeTok{na.rm=}\NormalTok{na.rm)}
\NormalTok{  SD <-}\StringTok{ }\KeywordTok{sd}\NormalTok{(x, }\DataTypeTok{na.rm=}\NormalTok{na.rm)}
  
\NormalTok{  final <-}\StringTok{ }\KeywordTok{data.frame}\NormalTok{(N, }\StringTok{"Mean"}\NormalTok{=M, }\StringTok{"SD"}\NormalTok{=SD)}
  \KeywordTok{return}\NormalTok{(final)}
\NormalTok{\}}
\KeywordTok{important_statistics2}\NormalTok{(v1)}
\end{Highlighting}
\end{Shaded}

\begin{verbatim}
##    N Mean       SD
## 1 10  1.8 1.032796
\end{verbatim}

We will come back to this function and use it in some loops and see what
else we can do with it.

\end{frame}

\section{Anonymous Functions}\label{anonymous-functions}

\begin{frame}[fragile]{Anonymous Functions}

Sometimes it is not worth saving a function but want to use it,
generally within loops.

\begin{Shaded}
\begin{Highlighting}[]
\ControlFlowTok{function}\NormalTok{(x) }\KeywordTok{thing}\NormalTok{(that, you, want, it, to, do, with, x)}
\end{Highlighting}
\end{Shaded}

We will show a few of these in the looping section (although they are
identical in nature to named functions, they just aren't named)

\end{frame}

\section{Why Write Your Own?}\label{why-write-your-own}

\begin{frame}{Why Write Your Own?}

Several reasons exist.

\begin{itemize}
\tightlist
\item
  Looping
\item
  Adjusting output
\item
  Performing a special function
\item
  Other customization
\end{itemize}

We are going to talk about looping in depth.

\end{frame}

\begin{frame}[fragile]{Looping}

Writing your own functions for looping is very common and practical.

Loops are things that are repeated.

For example:

\begin{itemize}
\tightlist
\item
  We may want a certain statistic (like a mean) for every continuous
  variable in the data set.
\item
  We may want to remove \texttt{999} from every variable in a data set.
\item
  We may want to change variable types of certain variables across the
  whole data set.
\end{itemize}

\end{frame}

\begin{frame}[fragile]{Looping}

Examples in \texttt{R} include:

\begin{itemize}
\tightlist
\item
  \texttt{for} loops
\item
  the \texttt{apply} family of functions
\end{itemize}

\end{frame}

\begin{frame}[fragile]{For Loops}

\begin{Shaded}
\begin{Highlighting}[]
\ControlFlowTok{for}\NormalTok{ (i }\ControlFlowTok{in} \DecValTok{1}\OperatorTok{:}\DecValTok{10}\NormalTok{)\{}
  \KeywordTok{mean}\NormalTok{(data[, i])}
\NormalTok{\}}
\end{Highlighting}
\end{Shaded}

\end{frame}

\begin{frame}[fragile]{For Loops}

Another example:

\begin{Shaded}
\begin{Highlighting}[]
\KeywordTok{library}\NormalTok{(tidyverse)}
\NormalTok{data =}\StringTok{ }\KeywordTok{read.csv}\NormalTok{(}\StringTok{"~/Dropbox/Teaching/R for Social Sciences/Data/WideFormat_TheOffice.csv"}\NormalTok{) }\OperatorTok
\StringTok{  }\KeywordTok{select}\NormalTok{(Prod1, MentalApt, PhysApt, Income, Children, SubsUse, Ment1, Ment2)}
\NormalTok{thing =}\StringTok{ }\KeywordTok{list}\NormalTok{()}
\ControlFlowTok{for}\NormalTok{ (i }\ControlFlowTok{in} \DecValTok{1}\OperatorTok{:}\DecValTok{8}\NormalTok{)\{}
\NormalTok{  thing[[i]] =}\StringTok{ }\KeywordTok{cbind}\NormalTok{(}\KeywordTok{mean}\NormalTok{(data[, i]), }\KeywordTok{sd}\NormalTok{(data[, i]))}
\NormalTok{\}}
\end{Highlighting}
\end{Shaded}

\end{frame}

\begin{frame}[fragile]{For Loops}

\begin{Shaded}
\begin{Highlighting}[]
\NormalTok{thing}
\end{Highlighting}
\end{Shaded}

\begin{verbatim}
## [[1]]
##      [,1]     [,2]
## [1,]  3.2 1.146423
## 
## [[2]]
##      [,1]     [,2]
## [1,]  5.2 2.305273
## 
## [[3]]
##          [,1]     [,2]
## [1,] 5.733333 1.869556
## 
## [[4]]
##          [,1]     [,2]
## [1,] 53.33333 12.63027
## 
## [[5]]
##      [,1]      [,2]
## [1,]  0.4 0.7367884
## 
## [[6]]
##           [,1]      [,2]
## [1,] 0.2666667 0.4577377
## 
## [[7]]
##          [,1]     [,2]
## [1,] 13.86667 3.602909
## 
## [[8]]
##          [,1]    [,2]
## [1,] 6.066667 2.65832
\end{verbatim}

\end{frame}

\begin{frame}[fragile]{For loops}

I like for loops. They are easy to understand and fiddle with, after
some practice.

However, they used to be slow in \texttt{R} and so they have a bad
reputation.

\end{frame}

\begin{frame}[fragile]{The apply family}

There are several \texttt{apply} functions in \texttt{R} that do loops
for you.

\begin{itemize}
\tightlist
\item
  \texttt{apply()}
\item
  \texttt{sapply()}
\item
  \texttt{lapply()}
\item
  \texttt{tapply()}
\end{itemize}

\end{frame}

\begin{frame}[fragile]{sapply}

Produces a vector based on the function that it is repeating.

Both do the same thing here.

\begin{Shaded}
\begin{Highlighting}[]
\ControlFlowTok{for}\NormalTok{ (i }\ControlFlowTok{in} \DecValTok{1}\OperatorTok{:}\DecValTok{10}\NormalTok{)\{}
  \KeywordTok{mean}\NormalTok{(data[, i])}
\NormalTok{\}}
\KeywordTok{sapply}\NormalTok{(}\DecValTok{1}\OperatorTok{:}\DecValTok{10}\NormalTok{, }\ControlFlowTok{function}\NormalTok{(i) }\KeywordTok{mean}\NormalTok{(data[, i]))}
\end{Highlighting}
\end{Shaded}

\end{frame}

\begin{frame}[fragile]{sapply}

Can also just provide the \texttt{data.frame} and it assumes you want
the function (in this case \texttt{mean()}) applied to each variable.

\begin{Shaded}
\begin{Highlighting}[]
\KeywordTok{sapply}\NormalTok{(data, mean)}
\end{Highlighting}
\end{Shaded}

\begin{verbatim}
##      Prod1  MentalApt    PhysApt     Income   Children    SubsUse 
##  3.2000000  5.2000000  5.7333333 53.3333333  0.4000000  0.2666667 
##      Ment1      Ment2 
## 13.8666667  6.0666667
\end{verbatim}

\end{frame}

\begin{frame}[fragile]{lapply}

Produces a list. Just like \texttt{sapply()} it takes a
\texttt{data.frame} and a function and applies it across the variables.

\begin{Shaded}
\begin{Highlighting}[]
\KeywordTok{lapply}\NormalTok{(data, mean)}
\end{Highlighting}
\end{Shaded}

\begin{verbatim}
## $Prod1
## [1] 3.2
## 
## $MentalApt
## [1] 5.2
## 
## $PhysApt
## [1] 5.733333
## 
## $Income
## [1] 53.33333
## 
## $Children
## [1] 0.4
## 
## $SubsUse
## [1] 0.2666667
## 
## $Ment1
## [1] 13.86667
## 
## $Ment2
## [1] 6.066667
\end{verbatim}

\end{frame}

\begin{frame}[fragile]{tapply}

Is a bit different than the rest. It doesn't do much in terms of looping
necessarily (although you can have it do that). Instead, it applies a
function based on a grouping variable. With the \texttt{tidyverse},
however, this is not often used any more.

\begin{Shaded}
\begin{Highlighting}[]
\KeywordTok{tapply}\NormalTok{(data}\OperatorTok{$}\NormalTok{Income, data}\OperatorTok{$}\NormalTok{Children, mean)}
\end{Highlighting}
\end{Shaded}

\begin{verbatim}
##        0        1        2 
## 53.18182 55.00000 52.50000
\end{verbatim}

\emph{How could you do this in the \texttt{tidyverse} framework?}

\end{frame}

\section{Loops with User-Defined
Functions}\label{loops-with-user-defined-functions}

\begin{frame}[fragile]{Loops with User-Defined Functions}

Going back to our \texttt{important\_statistics2()} function:

\begin{Shaded}
\begin{Highlighting}[]
\NormalTok{important_statistics2 <-}\StringTok{ }\ControlFlowTok{function}\NormalTok{(x, }\DataTypeTok{na.rm=}\OtherTok{FALSE}\NormalTok{)\{}
\NormalTok{  N  <-}\StringTok{ }\KeywordTok{length}\NormalTok{(x)}
\NormalTok{  M  <-}\StringTok{ }\KeywordTok{mean}\NormalTok{(x, }\DataTypeTok{na.rm=}\NormalTok{na.rm)}
\NormalTok{  SD <-}\StringTok{ }\KeywordTok{sd}\NormalTok{(x, }\DataTypeTok{na.rm=}\NormalTok{na.rm)}
  
\NormalTok{  final <-}\StringTok{ }\KeywordTok{data.frame}\NormalTok{(N, }\StringTok{"Mean"}\NormalTok{=M, }\StringTok{"SD"}\NormalTok{=SD)}
  \KeywordTok{return}\NormalTok{(final)}
\NormalTok{\}}
\end{Highlighting}
\end{Shaded}

Let's put it in a loop.

\end{frame}

\begin{frame}[fragile]{Loops with User-Defined Functions}

\begin{Shaded}
\begin{Highlighting}[]
\KeywordTok{lapply}\NormalTok{(data, important_statistics2)}
\end{Highlighting}
\end{Shaded}

\begin{verbatim}
## $Prod1
##    N Mean       SD
## 1 15  3.2 1.146423
## 
## $MentalApt
##    N Mean       SD
## 1 15  5.2 2.305273
## 
## $PhysApt
##    N     Mean       SD
## 1 15 5.733333 1.869556
## 
## $Income
##    N     Mean       SD
## 1 15 53.33333 12.63027
## 
## $Children
##    N Mean        SD
## 1 15  0.4 0.7367884
## 
## $SubsUse
##    N      Mean        SD
## 1 15 0.2666667 0.4577377
## 
## $Ment1
##    N     Mean       SD
## 1 15 13.86667 3.602909
## 
## $Ment2
##    N     Mean      SD
## 1 15 6.066667 2.65832
\end{verbatim}

\end{frame}

\begin{frame}[fragile]{Loops with User-Defined Functions}

\begin{Shaded}
\begin{Highlighting}[]
\KeywordTok{sapply}\NormalTok{(data, important_statistics2)}
\end{Highlighting}
\end{Shaded}

\begin{verbatim}
##      Prod1    MentalApt PhysApt  Income   Children  SubsUse   Ment1   
## N    15       15        15       15       15        15        15      
## Mean 3.2      5.2       5.733333 53.33333 0.4       0.2666667 13.86667
## SD   1.146423 2.305273  1.869556 12.63027 0.7367884 0.4577377 3.602909
##      Ment2   
## N    15      
## Mean 6.066667
## SD   2.65832
\end{verbatim}

\end{frame}

\begin{frame}[fragile]{Loops with User-Defined Functions}

It applied our function across the variables in the \texttt{data.frame}.
So we can easily get information we want.

This was a very simplified version of how I created the
\texttt{table1()} function in \texttt{furniture}.

\end{frame}

\section{Conclusions}\label{conclusions}

\begin{frame}[fragile]{Conclusions}

Writing your own functions takes time and practice but it can be a
worthwhile tool in using \texttt{R}.

I recommend you start simple and start soon.

Ultimately, you can make your own group of functions you use often and
create a package for it so others can use them too :)

\end{frame}
